\chapter{Abstract}

There are many phenotypic traits exhibited by animals which are useful to characterise but hard to measure. How cattle, and livestock in general, graze in free roaming pastures is one such phenotype of interest since it can be used to estimate the efficiency (feed to weight gain) of individual animals and thereby be selective bred for. Developing technologies to classify phenotype behaviours is a resource intensive and time consuming task due to the large amounts of sensory data that must be collected, annotated and presented in a manner suitable for classifier training.\\

This thesis has devised a Time-series Sensor Data Classifier Development Framework to allow the rapid development of phenotype classifiers and is demonstrated on the challenging problem of cattle chew classification. The framework allows large volumes of multi-sensor data to be quickly accessed and presented for annotation. This annotated data is then presented to a classifier trainer ultimately producing a classification algorithm which can then be ported to a device to allow detection of the phenotype trait in the real world. This framework will be used and built upon by CSIRO scientists.

