\chapter{Introduction}

The efficiency of livestock in converting feed to weight gained is an important metric for livestock managers. This metric allows for the identification of animals that can efficiently gain weight which can influence the livestock manager's selective breeding strategy in order to optimise economic meat production. 

Conversion of feed to red meat in livestock is inherently inefficient in terms of feed intake to weight gain. For beef, current estimates of the ratio of feed intake to weight gain are between 5:1 and 10:1 \cite{Garnett2009}. Opportunities to improve this efficiency through genetic controls such as breed substitution, crossbreeding and genetic selection exist \cite{Hill2012} but data that shows how efficient individual animals are at converting feed to meat is essential in knowing which livestock to breed. 

This data forms part of an animal's phenotype. The phenotype is the set of characteristics of an individual animal that are a result of both its genetics and environment. By measuring aspects of a particular animal's phenotype, information pertaining to how efficient that animal is at converting feed to weight can be found. This phenotype information can be used to extrapolate information about the genome which allows for the development of selection criteria for breeding objectives \cite{Pollak2012}. 

However, this specific livestock phenotype information is difficult to empirically obtain and there is currently a lack of data available. Current recording techniques have focused mainly on animals in feedlots, where the animals are kept in pens and fed \textit{ad libitum} at regulated times \cite{Arthur2005}.

This difficulty is compounded when considering livestock in pasture. Livestock in pasture feed directly from the pasture so it is not possible to regulate feeding times. Manually measuring feed intake of livestock in pasture is not practical as it would require significant manpower. Even in feedlots, where the process of measuring intake is easier because of the controlled way in which feeding is conducted, the cost of manpower is cited as the main reason for current lack of recording \cite{Barwick2010}. Furthermore, since heritability is both a property of the population and the environment \cite{Falconer1996}, measuring the traits of livestock in feedlots would not necessarily capture the traits of the same livestock in pasture. Attempts to capture this data through the use of chemical  markers to measure pasture intake in studies have had some success \cite{Barlow2009} \cite{Dove2006}; however, chemical markers have limitations and can only be used for short time periods. Therefore, to gain a robust understanding of the phenotype of individual animals, longer time periods of intake measurement are needed.

In recent decades there has been advancement in the development of small, low power sensor devices. These devices have allowed scientists to create non-invasive devices that can be used to collect movement traces of free-ranging animals \cite{Anthony2012}. It has been suggested that this data could be used for livestock behaviour classification \cite{Guo2006}; however developing a practical measure of intake for livestock in pasture currently remains a serious challenge \cite{Cottle2013}. Using these sensors as a method to automate the collective of phenotype information to assist in the process of selective breeding would be valuable to the livestock industry. Such a method would not only assist in decision making for selective breeding opportunities but could also assist in the management of livestock health since livestock managers would be able to overview livestock behaviour. 

The Australian Commonwealth Scientific and Industrial Research Organisation (CSIRO) have previously developed devices that can non intrusively fit onto livestock and collect signal data including inertial measurement and Global Positioning System (GPS) measurement data \cite{Guo2006}. There is a potential for these devices to be used for classifying livestock phenotypes. The aim of thesis is to explore the feasibility of using such sensors to develop an automated method for the collection of livestock phenotype information pertaining to pasture intake. 

The structure of this thesis is as follows:

\begin{itemize}
\item Chapter 2 contains a review of previous work in the areas of automating the classification of animal behaviour and discusses the impact of this work on the project. 

\item Chapter 3 provides information on relevant background theory and technologies to assist the reader in understanding the work performed in the thesis.

\item Chapter 4

\end{itemize}
