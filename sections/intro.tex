\chapter{Introduction}

The phenotype of an animal is the set of characteristics of an individual animal that are a result of both its genetics and environment. The complete phenotype of animal gives insight to traits of that animal. For example, the activity level or feed efficiency of an individual animal is information contained in its phenotype. 

For managing livestock towards economic or environmental goals, phenotypic traits can be useful to characterise in individual livestock. In selective breeding especially, phenotype information is critical in knowing which individual animals should be bread towards particular economic or environmental goals. \cite{Pollak2012} \cite{Arthur}

However, phenotypic traits can be hard and expensive to measure because, in order to do so, large amounts of data needs to be collected and analysed. For the accurate measurement of phenotypic traits, long periods of observation are needed in the commercial environment in which the animal is kept. \cite{Falconer1996}

In recent decades there has been advancement in the development of small, low power sensor devices. These devices have allowed scientists to create non-invasive devices that can be used to collect movement traces of free-ranging animals \cite{Anthony2012}. It has been suggested that this data could be used for automated livestock behaviour classification \cite{Guo2006} which would allow for measurement of phenotypic traits. 

Using these sensors as a method to automate the collection of phenotype information would be valuable to the livestock industry, however only preliminary trials have been attempted in this area \cite{Guo2006}. 

There is potential for techniques such as machine learning, signal processing or patten recognition to be used to develop such a phenotypic trait classifier \cite{Guo2006} and there is evidence to suggest that such a classifier would be able to run on a low-powered sensor device \cite{predd2007distributed} \cite{stoeltingmachine}. However, designing a classification algorithm is a complex process that requires collection, curation and annotation of large amounts of data as well as design and verification of the algorithm. 

This theis, titled, \textit{'Time-series Sensor Data Classifier Development Framework for Animal Phenotyping'} focused on simplifying the complex procedure of developing a classifier for phenotypic traits by developing a general use framework that would allow for the rapid development of automated, real-time phenotypic trait classifiers that could be implemented on a low powered sensor device. 

Although the framework was to be developed for any phenotypic trait measurement task, the development of the framework was driven and contextualised by the example phenotypic trait measurement task of cattle feed efficiency.

The structure of this thesis is broken down into the following sections:

\begin{itemize}

\item Chapter 2 contains relevant background and theory. Background is given into the task of cattle feed effeciency as well as classification algorithm design. Additionally, information about the work already performed on the project is shown. Some theory is also given into technologies used in the development of the framework. 

\item Chapter 3 reviews related previous work and a discussion of the impact of that work to this thesis.

\item Chapter 4 builds off the impact of the previous work to give a project outline. The goals of the thesis are formally stated in this section.

\item Chapter 5 details the method of creating the framework and gives insight to how each part of the framework fits together. 

\item Chapter 6 critically analyses the final version of the framework, and discusses the use of the framework for the cow feed effeciency analysis task. Future directions of the framework in general are also discussed.

\item Chapter 7 forms conclusions from the above sections.

\end{itemize}