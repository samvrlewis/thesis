\chapter{Project outline}
It has previously been found that there is a good relationship between the intake of digestible organic matter and gain in cattle \cite{Lippke1980}. Extending this, we postulate that there is a correlation between the amount of chewing events (ie individual chews or bites) and the amount of intake. Therefore, rather than trying to exactly quantify the intake of pasture in livestock the project will aim to quantify the amount of chewing events. 

The postulated correlation between chewing events and amount of intake can be experimentally confirmed. By weighing groups of livestock individually before and after a trial a relationship between the amount of time the animal spent feeding and the weight it gained can be confirmed. 

The main goal of the project is therefore automating the quantification of the amount of chewing events in individual livestock. As previously stated, this information can assist in the economic goals of livestock management by creating selective breeding opportunities. However, in order to achieve this goal, infrastructure first needs to be designed to facilitate this process. 


\section{Overall System Design}

Using the data recorded in the experiments outlined in section~\ref{sec:prior work}, the main goal of this project is to develop a classifier that can run on the sensor node eartag platform and quantify the amount of time individual cows spend feeding. Although the result of this work will be specific to cattle, it is envisioned that the technique will be transferable to other species of livestock. 

The desired outcome of this thesis was to design and implement a real time classifier to run on a wireless sensor node platform. However, it was first necessary to design infrastructure that facilitated the implementation and design of such a classifier. 

