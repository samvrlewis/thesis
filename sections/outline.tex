\chapter{Project outline}
It has previously been found that there is a good relationship between the intake of digestible organic matter and gain in cattle \cite{Lippke1980}. Extending this, we postulate that there is a correlation between the amount of chewing events (ie individual chews or bites) and the amount of intake. Therefore, rather than trying to exactly quantify the intake of pasture in livestock the project will aim to quantify the amount of chewing events. 

The postulated correlation between chewing events and amount of intake can be experimentally confirmed. By weighing groups of livestock individually before and after a trial a relationship between the amount of time the animal spent feeding and the weight it gained can be confirmed. 

The main goal of the project is therefore automating the quantification of the amount of chewing events in individual livestock. As previously stated, this information can assist in the economic goals of livestock management by creating selective breeding opportunities. However, in order to achieve this goal, a framework first needs to be designed to facilitate this process of designing a chew classification system.  

\newpage

\section{Classifier Design Process}

Using the data recorded in the experiments outlined in section~\ref{sec:prior work}, the main goal of this project is to develop a classifier that can run on the sensor node eartag platform and quantify the amount of time individual cows spend feeding. Although the result of this work will be specific to cattle, it is envisioned that the technique will be transferable to other species of livestock. 

\subsection*{Data Curation/Extraction}
As outlined in  section~\ref{sec:prior work}, a number of experiments with sensors attached to cattle have been performed. This data will be used to both design a classifier and verify the performance of it. 

The eartag stores data in the Tagged Data Format (TDF - see section~\ref{TDF} for more details) on to SD storage. Because of the large amount of sensor data collected experimentally, it would be convenient to curate the data so it is easier to access. Presently, the data is stored in a binary format; the first task of the project will be extracting and then converting the data into the Hierarchical Data Format (HDF). HDF is designed for use with large datasets of numerical data so it is perfect for the task.

Once the data has been converted into the HDF format it can be accessed through the RESTful Application Programming Interface (API). This allows for simple remote data access via a web address. The RESTful API also allows for data querying and simple statistics which will simplify the process of extracting data.

As soon as the data is able to be accessed through via the RESTful interface, the next step will be to investigate the data to find behavioural events for classification. This will involve finding when there is overlap between sensor data recordings and video so that the sensor data can be annotated with behavioural information.

\subsection*{Data Annotation}

Creating training and test sets is important for the design of a behaviour classifier. These sets consist of sets of input data and correct classifications for that input data. In this case, the sets will consist of chewing/not chewing events with correspoding sensor data. 

To create the sets, the videos of the cattle will be viewed and events will be found. The recorded audio will also be used to precisely match the event start and end times.  

To assist with this a data annotation interface  will be built. This should take the form of an application that allows sensor data to be loaded in and displayed so that a user can manually classify sections of the data as a chew/not chew event. Sections of the audio data should also be able to be played from this interface. 


For a good classifier, it is important to create sets that not only have positive events (eg. chewing) but also negative events (eg. any event where the animal is not chewing).

\subsection*{Classifier Design} 

\subsection*{Classifier Verification}


